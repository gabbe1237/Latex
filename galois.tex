\documentclass{article}
\DeclareMathAlphabet{\mathpzc}{OT1}{pzc}{m}{it}
\setlength{\parindent}{0pt}
\usepackage[table]{xcolor}
\usepackage{graphicx}
\usepackage{amssymb}
\usepackage{amsmath}
\usepackage{amsthm}
\usepackage{hyperref}
\usepackage{pgfplots}
\usepackage{pst-plot}
% \usepackage{fdsymbol}
\usepackage{empheq}
\usepackage{tikz}
\usepackage[most]{tcolorbox}
\usepackage{enumerate}
\usepackage{scalerel}
\usepackage{relsize}
\usepackage{mdframed}
\usepackage[utf8]{inputenc}

\usetikzlibrary{calc,patterns,angles,quotes}
\definecolor{myblue}{rgb}{2.55, 1.50, 0.20}
% \definecolor{myblue}{RGB}{227, 234, 253}
\newtcbtheorem[number within=subsection]{mytheo}{Sats}%
    {colback=myblue!15,colframe=black!30!black,
     enhanced,
     coltitle=black!75!black, boxrule=0.8pt,
     attach boxed title to top left=
       {xshift=2ex,yshift=-2mm,yshifttext=-1mm},
     boxed title style={colframe=black!30!black, boxrule=0.8pt,
       colback=myblue!60}}{th}

\newtcbtheorem[number within=subsection]{mydef}{Definition}%
    {colback=myblue!15,colframe=black!30!black,
     enhanced,
     coltitle=black!75!black, boxrule=0.8pt,
     attach boxed title to top left=
       {xshift=2ex,yshift=-2mm,yshifttext=-1mm},
     boxed title style={colframe=black!30!black, boxrule=0.8pt,
colback=myblue!60}}{th}

\newtcbtheorem[number within=subsection]{mykol}{Korollarium}%
    {colback=myblue!15,colframe=black!30!black,
     enhanced,
     coltitle=black!75!black, boxrule=0.8pt,
     attach boxed title to top left=
       {xshift=2ex,yshift=-2mm,yshifttext=-1mm},
     boxed title style={colframe=black!30!black, boxrule=0.8pt,
colback=myblue!60}}{th}

\newtcbtheorem[number within=subsection]{mylemma}{Lemma}%
    {colback=myblue!15,colframe=black!30!black,
     enhanced,
     coltitle=black!75!black, boxrule=0.8pt,
     attach boxed title to top left=
       {xshift=2ex,yshift=-2mm,yshifttext=-1mm},
     boxed title style={colframe=black!30!black, boxrule=0.8pt,
colback=myblue!60}}{th}

% ---------------------------------------------
% ---------------RE-NEW COMMAND----------------
% ---------------------------------------------
\newcommand\mul[1]{\multicolumn{1}{c}{#1}}
\setlength{\parskip}{1em}
\renewcommand{\baselinestretch}{1.2}
\renewcommand*{\proofname}{Bevis}
\newcommand{\ovning}[1]{\noindent {\bf Övning #1.}}
\renewcommand{\contentsname}{Innehåll}
\newcommand{\orbit}[0]{\mathlarger{\mathlarger{\mathcal{O}}}}
\newcommand{\grad}[0]{\textnormal{deg}}
\newtheorem{definition}{Definition}[section]
\pgfplotsset{compat=1.8}

\theoremstyle{definition}
\newtheorem{thm}{Theorem}[section]
\newtheorem{exmp}[thm]{Exempel}
\begin{document}
\title{%
  Galoisteori och de olösbara polynomen \\
  \small{En introduktion till det abstrakta - Del II}
}
\author{Gabriel Rajkowski}
\date{\today}

\maketitle

\thispagestyle{empty}

\clearpage
\tableofcontents
\section{Inledning}
\section{Historia}
\section{Komplement till del I}
Denna del ägnar sig åt saker som inte togs upp i del I på grund av de inte fick plats eller som inte var nödvändiga men som kommer att komma 
till hands i denna del.
\subsection{Linjär algebra}
En stor sak som undveks i del I var vektorrum. Det visar sig faktiskt att vektorrum blir väldigt praktiska och går hand i hand med andra 
algebraiska strukturer såsom kroppar och dess utvidgningar. I detta delavsnitt kommer vi att studera grunderna för vektorrum så att de senare kan användas till 
våran fördel.

\begin{mydef}{Vektorrum}{}
  Ett vektorrum över en kropp (vi definerar detta senare) $F$ är en mängd $V$ under addition och subtraktion som uppfyller följande axiom.
  \begin{enumerate}[I)]
    \item $V$ skapar en abelsk grupp under addition,
    \item $a \cdot (b \cdot v) = (a \cdot b) \cdot v$,
    \item $1 \cdot v = v$ där 1 är den multiplikativa enhetselementet i $F$,
    \item $a \cdot (u + v) = (a \cdot u) + (a \cdot v)$,
    \item $(a + b) \cdot v = (a \cdot v) + (b \cdot v$),
  \end{enumerate}
  där $a, b \in F$ och $v, u \in V$.
\end{mydef}
Elementen i $F$ brukar kallas för skalärer och elementen i $v$ 
brukar kallas för vektorer. 

% \begin{mydef}{Linjära beroenden}{}
%   Låt $v_1, v_2, \cdots, v_n$ vara element i ett vektorrum 
%   $V$ över en kropp $F$ och låt $a_1, a_2, \cdots, a_n \in F$.
%   Vektorerna är \textit{linjärt oberoende} om ekvationen 
%   \[a_1v_1 + a_2v_2 + \cdot + a_nv_n = 0\]
%   endast har lösningen $a_1 = a_2 = \ldots a_n = 0$.
%   Om sådant inte är fallet, det vill säga då 
%   en vektor kan skrivas som en linjär kombination av de andra, 
%   så är vektorerna \textit{linjärt beroende}.
% \end{mydef}
(några exempel)

\begin{mydef}{Baser}{}
  En delmängd $K$ av ett vektorrum $V$ över en kropp $F$ är en \textit{bas} för $V$ om följande villkor uppfylls.  
  \begin{enumerate}[I)]
    \item $K$ är \textit{linjärt oberoende}. För $v_1, \ldots, v_n \in K$ och $a_1, \ldots, a_n \in F$ så har ekvationen 
    \[a_1v_1 + \cdots + a_n v_n = 0\]
    lösningen $a_1 = \cdots = a_n = 0$.
    \item $K$ \textit{spänner upp} $V$. För alla $v \in V$ så existerar det $b_1, \ldots b_m \in F$ och $v_1, \ldots v_m \in K$ så att 
    $v = b_1 v_1 + \cdots b_m v_m$.
  \end{enumerate}
\end{mydef}

\subsection{Gruppteori}

\section{Ringar, kroppar och polynom}
\subsection{Ringar}
Ha kanske med denna del om den är nödvändig i framtiden. 
\subsection{Kroppar}
För att börja med Galoisteori så måste vi först ha en plats där vi kan utföra vanlig aritmetik, en \textit{kropp}. Galoisteori kan ses som en studie 
av dessa platser som tillåter grundläggande aritmetik och dess relation till polynom. Vi skall i detta delavsnitt och i avsnittet "kroppsutvidgningar" 
presentera de huvudsakliga faktan om 
sådana strukturer. 

\begin{mydef}{Kroppar}{}
    En \textit{kropp} är en mängd $F$ under två binära operationer, addition ($+$) och multiplikation ($\cdot$), som uppfyller följande \textit{kroppsaxiom}.
    \begin{enumerate}[I)]
        \item \textbf{Assosiativitet.} $a + (b + c) = (a + b) + c$ och $a \cdot (b \cdot c) = (a \cdot b) \cdot c$.
        \item \textbf{Kommutativitet.} $a + b = b + a$ och $a \cdot b = b \cdot a$.
        \item \textbf{Additativa och multiplikativa enhetselementet.} Det existerar två olika element $0$ och $1$ så att $a + 0 = a$ och $a \cdot 1 = a.$
        \item \textbf{Additativ invers.} För alla $a$ så existerar det en \textit{additativ invers} till $a$, som betecknas
        $-a$, så att $a + (-a) = 0$.
        \item \textbf{Multiplikativ invers.} För alla $a \neq 0$ så existerar det en \textit{multiplikativ invers} till $a$, som betecknas $a^{-1}$ eller $1/a$, så att 
        $a \cdot a^{-1} = 1.$
        \item \textbf{Distributivitet av multiplikation över addition.} $a \cdot (b + c) = (a \cdot b) + (a \cdot c)$,
    \end{enumerate}
    för $a, b, c \in F$
\end{mydef}
En kropp skapar alltså en abelsk grupp under addition och kroppen utom $0$ skapar en abelsk grupp under multiplikation. Alternativt kan en kropp sägas
vara en kommutativ ring. Om $K$ är en kroppsutvidgning till $F$ så kommer vi säga att $F$ är baskroppen. 


\begin{mydef}{Karakteristik}{}
  Låt $F$ vara en kropp. Om det existerar ett minsta positiva tal $n$ så att $\underbrace{1 + \cdots + 1}_{n} = 0$ så kallas $n$ för kroppens \textit{karakteristik}.
  Om $n$ inte existerar är karakteristiken för kroppen $0.$
\end{mydef}
Man kan göra en koppling mellan karakteristik och cykliska grupper där alla element genereras genom ett enda element. Om en kropps karakteristik inte är noll
så kan den anses vara "cyklisk". 

\subsection{Polynom}
\begin{mydef}{Polynom}{}
  Låt $F$ vara en kropp. Ett \textit{polynom} över $F$ är en ekvation på följande form
  \[f(X) = a_0 + a_1X + a_2X^2 + \cdots a_nX^n\]
  där $n \in \mathbb{N}$ och där \textit{koefficienterna} $a_1, a_2, \ldots, a_n \in F$. 
  Mängden $F[X]$ består av alla polynom vars koeffcienter tillhör $F$. \textit{Graden} för ett polynom är den högsta multipliciteten av den obestämda variablen som förekommer i  
  polynomet och betecknas $\grad f(X)$. (För polynomet ovan är $\grad f(X) = n$.)
\end{mydef}
Trots att $\mathbb{Z}$ inte är en kropp så kommer vi fortfarande beteckna $\mathbb{Z}[X]$ som mängden av polynom med heltals koeffcienter
för att slippa skriva mycket.

\begin{mydef}{Irreducibla polynom}{}
  Låt $f(X)$ vara ett polynom över kroppen $F$. Vi säger att $f(X)$ är \textit{irreducibelt} över $F$ om det inte kan faktoriseras som $f(X) = g(X)h(X)$
  för $g(X), h(X) \in F[X]$ vars grad är mindre än $f(X)$. Om faktoriseringen går att åstadkomma säger vi att $f(X)$ är \textit{reducibelt} över $F.$
\end{mydef}
Notera att denna faktorisering beror starkt på vilken kropp en jobbar med. I till exempel $\mathbb{Q}[X]$ finns det endast några "få" polynom som är reducibla
över $\mathbb{Q}[X]$, medan alla polynom över $\mathbb{C}[X]$ är reducibla över $\mathbb{C}[X]$. Det sistnämnda är algebrans fundemental sats och 
bevisas senare i texten.

Nedan följer några exempel på metoder för att visa att polynom är irreducibla. 
(några övningar eller exempel, beroende på vad jag väljer)

\begin{mylemma}{Gauss's lemma}{}
Låt $f(X)$ vara ett polynom med heltals koeffcienter. Då är $f(X)$ irreducibel över $\mathbb{Z}$ om och endast om den är irreducibel över $\mathbb{Q}$.
\end{mylemma}
\begin{proof}
  Det som skall bevisas är följande
  \begin{align*}
    f(X) \textnormal{ är irreducibel över } \mathbb{Z} &\Rightarrow f(X) \textnormal{ är irreducibel över } \mathbb{Q}, \\
    f(X) \textnormal{ är irreducibel över } \mathbb{Q} &\Rightarrow f(X) \textnormal{ är irreducibel över } \mathbb{Z}.
  \end{align*}
  Påståendet är ekvivalent med det kontrapositiva påståendet.
  \begin{align}
    f(X) \textnormal{ är irreducibel över } \mathbb{Q} &\Rightarrow f(X) \textnormal{ är irreducibel över } \mathbb{Z}, \\
    f(X) \textnormal{ är irreducibel över } \mathbb{Z} &\Rightarrow f(X) \textnormal{ är irreducibel över } \mathbb{Q}.
  \end{align}
  Påstående (2) är trivial, då $\mathbb{Z}[X]$ är en delmängd till $\mathbb{Q}[X]$ (ett heltal är även ett rationellt tal). Vi bevisar påstående (2),
  att om $f(X)$ är reducibel över $\mathbb{Q}$ så medför detta att den också är reducibel över $\mathbb{Z}$. Anta att
  \begin{equation}
    f(X) = g_1(X)g_2(X)
  \end{equation}
  där $g_1(X), \; g_2(X) \in \mathbb{Q}[X]$. Eftersom koefficienterna för $g_1(X)$ och $g_2(X)$ är rationella tal så kan vi multiplicera med ett
  heltal $n$ i båda led i (3) som delar alla koefficienternas nämnare hos $g_1(X)$ och $g_2(X)$. Vi får då ekvationen
  \begin{equation}
    nf(X) = h_1(X)h_2(X)
  \end{equation}
  där $h_1(X), \; h_2(X) \in \mathbb{Z}[X]$. Bland alla dessa uttryck, välj $n$ till det minsta positiva heltalet
  så att faktoriseringen går att utföra. Vi påstår att $n = 1$ som då kommer att slutföra beviset. Vi antar motsatsen, att $n$ inte är 1. Eftersom 
  $n$ delar alla nämnare hos $g_1(X)$ och $g_2(X)$ så kan det inte vara ett primtal och därför måste ett primtal dela $n$. Kalla detta primtal för $p$, då 
  måste även $p$ dela högerledet, $h_1(X)h_2(X)$, eftersom $p$ delar vänsterledet. Anta att 
  \[h_1(X) = a_0 + a_1X + \cdots + a_lX^l, \quad h_2(X) = b_0 + b_1X + \cdots + b_mX^m\]
  för Vi påstår att $p$ antigen delar alla $a_i$ eller alla $b_i$. Antag motsatsen. Det finns nu två olika motsatser som kan antas.
  \begin{enumerate}[I)]
    \item Antigen delar $p$ några $a_i$ \textit{eller} några $b_i$,
    \item $p$ delar några $a_i$ \textit{och} några $b_i$.
  \end{enumerate}
  Påstående I) är falsk då alla koeffcienter i $h_1(X)h_2(X)$ måste vara delbara med $p$ och om endast några koeffcienter i det ena polynomet 
  är delbara med $p$ men inte i den andra så uppfylls detta inte. 

  Vi antar att påstående II) gäller. Det är då lämpligt att anta att $p$ delar $a_0, a_{1}, \ldots, a_{j-1}$ men inte $a_j$ och att 
  $p$ delar $b_0, b_{1}, \ldots, b_{k-1}$ men inte $b_k$. Observera att
  \[c_{j+k} = a_0b_{j+k} + \cdots + a_{j-1}b_{k+1} + a_jb_k + a_{j+1}b_{k-1} + \cdots + a_{j+k}b_0\]
  är en koeffcient i $f(X)$ och $p$ måste därför dela $c_{j+k}$. Vi vet även att $p$ delar $a_0, a_{1}, \ldots, a_{j-1}, b_0, b_{1}, \ldots, b_{k-1}$ 
  och därför måste $p$ dela $a_j b_k$. Detta leder till en motsägelse eftersom $p$ måste isåfall dela $a_j$ och/eller $b_k$ och vi drar slutsatsen att 
  $p$ delar alla koeffcienter i $h_1(X)$ eller alla koeffcienter i $h_2(X)$. Låt oss anta det förra. Definera nu 
  \[h_1'(X) = \frac{1}{p} h_1(X)\]
  som, enligt det ovanstående antagandet, tillhör $\mathbb{Z}[X].$ Vi kan därför dela (4) med $p$ för att få
  \[\frac{n}{p} f(X) = h_1'(X) h_2(X).\]
  Detta strider mot antagandet att $n$ är minimal och vi kan konstatera att $n=1$. Detta slutför beviset för lemmat.
\end{proof}

\begin{mytheo}{Eisensteins kriterium}{}
  Låt
  \[f(X) = a_0 + a_1X + \cdots + a_nX^n\]
  vara ett polynom över $\mathbb{Z}$. Om det existerar ett primtal $p$ så att 
  \begin{enumerate}[I)]
    \item $p$ inte delar $a_n$,
    \item $p$ delar $a_0, a_1, \ldots, a_{n-1}$,
    \item $p^2$ inte delar $a_0$,
  \end{enumerate}
  så är $f(X)$ irreducibel över $\mathbb{Q}$.
\end{mytheo}
\begin{proof}
  Enligt Gauss's lemma räcker det med att visa att $f(X)$ är irreducibel över $\mathbb{Z}$. Anta att $f(X) = g_1(X) g_2(X)$ där 
  \[g_1(X) = b_0 + b_1X + \cdots b_kX^k, \quad g_2(X) = c_0 + c_1X + \cdots c_lX^l\]
  för $b_0, b_1, \ldots b_k, c_0, c_1, \ldots, c_l \in \mathbb{Z}.$
  Notera att $a_0 = b_0 c_0$ och därför, enligt II) och III), så måste $p$ dela $b_0$ eller $c_0$ men inte både och. Låt oss anta, utan förlust av 
  generalitet, att $p$ delar $b_0$ men inte $c_0$. Enligt I) så kan $p$ inte dela alla koeffcienter i $g_1(X)$. Detta betyder att det måste existera ett $b_i$ så att $p$ delar
  $b_0, b_1, \ldots, b_{i-1}$ men inte $b_i$ för $i \leq k < n$. Enligt II) har vi då att 
  \[a_i = b_0c_i + b_1c_{i-1} + \cdots + b_{i-1}c_1 + b_ic_0\]
  är delbart med $p$. Detta medför att $p$ delar $b_i c_0$, vilket är omöjligt enligt tidigare antaganden.
  Denna motsägelse slutför beviset.
\end{proof}
(några bra övningar)

\section{Kroppsutvidgningar}
\begin{mydef}{Kroppsutvidgningar}{}
  Låt $F$ och $K$ vara kroppar. Om $F$ är en delmängd till $K$ så är $K$ en \textit{kroppsutvidgning} till $F$ och vi skriver antigen $F \subset K$ eller $K/F$.
\end{mydef}
Denna definition vrider lite på perspektivet. Vanligtvist brukar man börja med en mängd och "jobba sig ner" till något mindre, en delmängd. Här är filosofin bakom detta lite 
annorlunda, man börjar med en mängd för att "jobba sig uppåt" till något större. 

Det första som skall observeras är följande. I dessa punkter är $K$ en kroppsutvidgning till $F$ och $a, b \in F$ samt $x, y \in K$.

\begin{enumerate}[I)]
  \item $K$ skapar en abelsk grupp under addition (det är en kropp),
  \item $a \cdot (b \cdot x) = (a \cdot b) \cdot x$,
  \item $1 \cdot x = x$,
  \item $a \cdot (x + y) = (a \cdot x) + (a \cdot y)$,
  \item $(a + b) \cdot x = (a \cdot x) + (b \cdot x)$.
\end{enumerate}
En kroppsutvidgning kan alltså ses som ett vektorrum över baskroppen. 

\begin{mydef}{Graden för en kroppsutvidgning}{}
  \textit{Graden} för en kroppsutvidgning $K$ till $F$ är kardinaliteten (storleken) av basen för $K$ och betecknas $[K:F]$. Om $[K:F]$ är ändlig så 
  säger vi att $K$ är en \textit{ändlig kroppsutvidgning} till $F$. 
\end{mydef}
Notera att en ändlig kroppsutvidgning inte betyder att kardinaliteten av själva kroppsutvidgningen är ändlig. En ändlig kroppsutvidgning
tyder endast på att kardinaliteten av basen, eller \textit{dimensionen} för vektorrumet som det också kallas, är ändlig. Nedan följer exempel på detta.
(några bra exempel som förtydligar detta.) 

\begin{mytheo}{Torn lagen (eng. tower law)}{}
  Anta att $F, \; K$ och $L$ bildar ett "torn" av kroppsutvidgningar, $F \subseteq K \subseteq L$. Då gäller
  \[ [L:F] = [L:K] \cdot [K:F]. \]
\end{mytheo}
\begin{proof}
  Vi antar att $[L:K]$ och $[K:F]$ är ändliga. Låt $\{v_1, \ldots, v_m\}$ vara basen för $L$ över $K$ och $\{w_1, \ldots, w_n\}$ vara basen för $K$ över $F$.
  Vi påstår att $B = \{v_iw_j \; | \; 1 \leq i \leq m, \; 1 \leq j \leq n\}$ är basen för $L$ över $F$ som då slutför beviset.

  Eftersom $\{v_1, \ldots, v_m\}$ är basen för $K$ över $F$ så spänner den upp $K$ och därför gäller, för $x \in L$ och $a_1, \ldots, a_m \in K$, att 
  \begin{equation}
    x = \sum_{i = 1}^m a_i v_i.
  \end{equation}
  Vi använder nu faktumet att $\{w_1, \ldots, w_n\}$ spänner upp $L$ för att få, för 
  \linebreak
  $b_{1i}, \ldots, b_{ni} \in F$,
  \begin{equation}
    a_i = \sum_{j = 1}^n b_{ji}w_j.
  \end{equation}
  Ekvation (6) i (5) mynnar ut i att
  \[x = \sum_{i = 1}^m \sum_{j = 1}^n b_{ji}w_jv_i\]
  och vi drar slutsatsen att $B$ spänner upp $L$ över $F$. Anta nu det existerar ett $c_ji \in F$ så att 
  \[\sum_{i = 1}^m \sum_{j = 1}^n c_{ji}w_jv_i = 0.\]
  Vi skriver om detta som 
  \[\sum_{i = 1}^m \biggl( \sum_{j = 1}^n c_{ji}w_j \biggl) v_i = 0\]
  och använder informationen att 
\end{proof}


\section{Galoisteorins fundamentalsats}
\section{Lösbara grupper och lösningarna till polynom}
\end{document}